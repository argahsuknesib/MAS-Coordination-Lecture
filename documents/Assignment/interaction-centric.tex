\documentclass[]{article}

%opening
\title{Interaction-Centric Coordination}
\author{Kushagra Singh BISEN}

\begin{document}

\maketitle

\section{Coordination by Conversation}

\subsection{Questions}

\subsubsection{What are the hidden hypothesis of coordination by messages like in this scenario?}
In the previous section of the lecture, we studied agent-dimension based coordination. By coordinating using 
messages,  
\subsubsection{Discuss the pros and cons of this approach versus the agent-centric coordination approach.}
\subsubsection{What is missing in the description of coordination by messages? Why?}

\subsection{Management of delay for performing a task}

\subsubsection{Deduce the semantic of the performatives tell, achieve, and askOne, and define when they should be used.}
After carrying out Q3, when you use the 'tell' method, the sender will make that the receiver will add the content 
of the message sent into it's belief base. For example, after sending \emph{.send(giacomo, tell, delay(plumbing,10)).}, when
you go to the Jason HTTP server, you find the giacomo agent having a belief base added with it's source \emph{delay(plumbing,10)[source(agenttest)].}

In Q4, we use the 'achieve' method, where the receiver agent will add the content of the message as a goal \emph{'to acheive'}.
After carrying out the \emph{.send(giacomo, achieve, updateTolerance(10)).} message in agenttest REPL agent, the goal
\emph{!updateTolerance(10)[source(agenttest)]} will be added. Moreover, as there is no existing plan to achieve this goal, 
we will have a message for the same in our giacomo console.

In Q5, when we use the 'askOne' method, the sender agent wishes to know and confirm if there is something like that is present
in the receiver agent's belief base by specifying the method and arguments. After carrying out, \emph{send(giacomo, askOne, penalty(X,Y)).} we 
wish to know the penalty method of the receiver agent if there is one. In response, we get \emph{penalty(plumbing,50)[source(giacomo)].}
in the agenttest's belief base.

Thus, we can conclude that the semantics of the performatives for the methods are completely different.
\section{Protocol Definition}

\subsection{Questions}
\subsubsection{What are the pros and cons of coordination with protocol?}


\end{document}